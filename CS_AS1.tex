\documentclass[aspectratio=43]{beamer}

\mode<presentation>
\usepackage{amsmath}
\usepackage{amssymb}

\usepackage{graphicx} % takes care of graphic including machinery
\usepackage[american]{circuitikz}
\usepackage{pgfplots}
\pgfplotsset{width=10cm,compat=1.9}
%\usepackage{advdate}
\usepackage{times}
\usepackage{adjustbox}
\usepackage{subcaption}
\usepackage{enumitem}
\usepackage{multicol}
\usepackage{listings}
\usepackage{url}
\def\UrlBreaks{\do\/\do-}
\usetheme{Copenhagen} 
\usecolortheme{seahorse}
\setbeamertemplate{footline}
{
  \leavevmode%
  \hbox{%
  \begin{beamercolorbox}[wd=\paperwidth,ht=2.25ex,dp=1ex,right]{author in head/foot}%
    \insertframenumber{} / \inserttotalframenumber\hspace*{2ex} 
  \end{beamercolorbox}}%
  \vskip0pt%
}
\setbeamertemplate{navigation symbols}{}

\providecommand{\nCr}[2]{\,^{#1}C_{#2}} % nCr
\providecommand{\nPr}[2]{\,^{#1}P_{#2}} % nPr
\providecommand{\mbf}{\mathbf}
\providecommand{\pr}[1]{\ensuremath{\Pr\left(#1\right)}}
\providecommand{\qfunc}[1]{\ensuremath{Q\left(#1\right)}}
\providecommand{\sbrak}[1]{\ensuremath{{}\left[#1\right]}}
\providecommand{\lsbrak}[1]{\ensuremath{{}\left[#1\right.}}
\providecommand{\rsbrak}[1]{\ensuremath{{}\left.#1\right]}}
\providecommand{\brak}[1]{\ensuremath{\left(#1\right)}}
\providecommand{\lbrak}[1]{\ensuremath{\left(#1\right.}}
\providecommand{\rbrak}[1]{\ensuremath{\left.#1\right)}}
\providecommand{\cbrak}[1]{\ensuremath{\left\{#1\right\}}}
\providecommand{\lcbrak}[1]{\ensuremath{\left\{#1\right.}}
\providecommand{\rcbrak}[1]{\ensuremath{\left.#1\right\}}}
\theoremstyle{remark}
\newtheorem{rem}{Remark}
\newcommand{\sgn}{\mathop{\mathrm{sgn}}}
\providecommand{\abs}[1]{\left\vert#1\right\vert}
\providecommand{\res}[1]{\Res\displaylimits_{#1}} 
\providecommand{\norm}[1]{\lVert#1\rVert}
\providecommand{\mtx}[1]{\mathbf{#1}}
\providecommand{\mean}[1]{E\left[ #1 \right]}
\providecommand{\fourier}{\overset{\mathcal{F}}{ \rightleftharpoons}}
%\providecommand{\hilbert}{\overset{\mathcal{H}}{ \rightleftharpoons}}
\providecommand{\system}{\overset{\mathcal{H}}{ \longleftrightarrow}}
	%\newcommand{\solution}[2]{\textbf{Solution:}{#1}}
%\newcommand{\solution}{\noindent \textbf{Solution: }}
\providecommand{\dec}[2]{\ensuremath{\overset{#1}{\underset{#2}{\gtrless}}}}
\newcommand{\myvec}[1]{\ensuremath{\begin{pmatrix}#1\end{pmatrix}}}
\let\vec\mathbf

\usepackage{xcolor}

%New colors defined below
\definecolor{codegreen}{rgb}{0,0.6,0}
\definecolor{codegray}{rgb}{0.5,0.5,0.5}
\definecolor{codepurple}{rgb}{0.58,0,0.82}
\definecolor{backcolour}{rgb}{0.95,0.95,0.92}

%Code listing style named "mystyle"
\lstdefinestyle{mystyle}{
  backgroundcolor=\color{backcolour},   commentstyle=\color{codegreen},
  keywordstyle=\color{magenta},
  numberstyle=\tiny\color{codegray},
  stringstyle=\color{codepurple},
  basicstyle=\ttfamily\footnotesize,
  breakatwhitespace=false,         
  breaklines=true,                 
  captionpos=b,                    
  keepspaces=true,                 
  numbers=left,                    
  numbersep=5pt,                  
  showspaces=false,                
  showstringspaces=false,
  showtabs=false,                  
  tabsize=2
}

%"mystyle" code listing set
\lstset{style=mystyle}

\numberwithin{equation}{section}

\title{Control System - Assignment\\Problem 20}
\author{S.V.Harshith - EE19BTECH11018}
\date{\today}
\begin{document}

\begin{frame}
\titlepage
\end{frame}


\begin{frame}
\frametitle{Question-(a)}
Write, but do not solve, the mesh and nodal
equations for the network of Figure given below. 

\includegraphics[width=0.7\columnwidth]{a.png}
\centering
\end{frame}
\begin{frame}

Assuming current $i_1(t)$ in the bottom left loop ,$i_2(t)$ in bottom right loop and $i_3(t)$ in top loop\\
Assuming Voltage $v_1(t)$ in the middle node.\\
After converting into laplace domain we get the below circuit -

\begin{circuitikz}[scale=1.2]\draw
(0,3) to[V, l=$V(s)$ , i =$I_1(s)$ ] (0,0)
(0,3) -- (0,4.5)
to[C, l=$\frac{9}{s}$,i=$I_3(s)$ ] (5,4.5) -- (5,2)
to[R, l=8,v=$V_0(s)$] (5,0) -- (0,0)
(0,3) to[R, l=2] (2,3)node[label={$V_1(s)$}]
(2.5)to[R, l=4$\ohm$] (3.5,3)
to[L, l=6s] (5,3)
(2,0) to[L , l = 4s] (2,1.5)
to [R , l =2](2,2.5) -- (2,3)
(5,3) to[short,i = $I_2(s)$](5,2)
;\end{circuitikz}
\centering
\end{frame}
\begin{frame}
\frametitle{Mesh equations}
Using Mesh analysis we get the below equations -
\begin{equation*}
(4 + 4s)I_1(s) - (2 + 4s)I_2(s) - 2I_3(s) = V(s)
\end{equation*}
\begin{equation*}
-(2 + 4s)I_1(s) + (14 + 10s)I_2(s) - (4+6s)I_3(s) = 0
\end{equation*}
\begin{equation*}
-2I_1(s) - (4 + 6s)I_2(s) + (6 + 6s+ \frac{9}{s})I_3(s) = 0
\end{equation*}
\end{frame}
\begin{frame}
\frametitle{Nodal equations}

Using Nodal analysis we get the below equations -
\begin{equation*}
\frac{(V_1(s) - V(s))}{2}+\frac{V_1(s)}{2+4s}+\frac{(V_1(s) - V_0(s))}{4+6s} = 0
\end{equation*}
\begin{equation*}
\frac{(V_0(s) - V_1(s))}{4+6s}+\frac{V_0(s)}{8}+\frac{(V_0(s) - V(s))}{\frac{9}{s}} = 0
\end{equation*}

\end{frame}
\begin{frame}
\frametitle{Question-(b)}
Use Python, and the equations found in part a to solve for the transfer function, $G(s) = V_0(s)/V(s)$. Use both the mesh and nodal equations and show that either set yields the same transfer function
\end{frame}
\begin{frame}
\frametitle{Python code}
\lstinputlisting[language=Python
]{controlsys.py}
\end{frame}
\begin{frame}
\frametitle{Output}
We can see in the terminal below that the transfer functions obtained from both mesh and nodal equations are same
\includegraphics[width=1\columnwidth]{CSp2.png}
\end{frame}
\end{document}


